\week{16}{13 Jun.\ 2024}{Tighter Bounds on Sub-Gaussian Matrices}
\section{Two-sided bounds on sub-gaussian matrices}
\begin{problem*}[Exercise 4.6.2]\label{ex4.6.2}
	Deduce from (4.22) that
	\[
		\mathbb{E}_{}\left[\left\lVert \frac{1}{m}A ^{\top} A - I_n \right\rVert \right]
		\leq CK^2 \left( \sqrt{\frac{n}{m}} + \frac{n}{m} \right) .
	\]
\end{problem*}
\begin{answer}
	We have that for any \(t \geq 0\), with probability at least \(1 - 2 \exp (- t^2)\),
	\[
		\left\lVert \frac{1}{m}A^{\top} A - I_n \right\rVert
		\leq K^2 \max (\delta , \delta ^2), \text{ where }
		\delta = C \left( \sqrt{\frac{n}{m}} + \frac{t}{\sqrt{m} } \right),
	\]
	and we want to prove
	\[
		\mathbb{E}_{}\left[\left\lVert \frac{1}{m}A^{\top} A - I_n \right\rVert \right]
		\leq C K^2 \left( \sqrt{\frac{n}{m}} + \frac{n}{m} \right) .
	\]
	Firstly, we know that with \(u \coloneqq K^2 ((\frac{C}{\sqrt{m} } + \frac{2C^2 \sqrt{n} }{m})t + \frac{C^2}{m}t^2)\), we get exactly
	\[
		\mathbb{P} \left( \left\lVert \frac{1}{m} A^{\top} A - I_n \right\rVert > K^2 \left( C \sqrt{\frac{n}{m}} + C^2 \frac{n}{m} \right) + u \right)
		\leq 2 e^{-t^2}.
	\]
	Then, from the integral identity with the substitution \(v \coloneqq u + K^2 ( C \sqrt{\frac{n}{m}} + C^2 \frac{n}{m})\),
	\begin{align*}
		 & \mathbb{E}_{}\left[\left\lVert \frac{1}{m}A^{\top} A - I_n \right\rVert \right]                                                                                                                                                               \\
		 & = \left( \int_{0}^{K^2 (C \sqrt{\frac{n}{m}} + C^2 \frac{n}{m})} + \int_{K^2 (C \sqrt{\frac{n}{m}} + C^2 \frac{n}{m})}^{\infty} \right) \mathbb{P} \left( \left\lVert \frac{1}{m}A^{\top} A - I_n \right\rVert > v \right)  \,\mathrm{d}v     \\
		 & \leq \int_{0}^{K^2 (C \sqrt{\frac{n}{m}} + C^2 \frac{n}{m})} 1 \,\mathrm{d}v + \int_{K^2 (C \sqrt{\frac{n}{m}} + C^2 \frac{n}{m})}^{\infty} \mathbb{P} \left( \left\lVert \frac{1}{m}A^{\top} A - I_n \right\rVert > v \right)  \,\mathrm{d}v \\
		 & = K^2 \left( C \sqrt{\frac{n}{m}} + C^2 \frac{n}{m} \right) + \int_{0}^{\infty} \mathbb{P} \left( \left\lVert \frac{1}{m}A^{\top} A - I_n \right\rVert > K^2 \left( C \sqrt{\frac{n}{m}} + C^2 \frac{n}{m} \right) + u \right)  \,\mathrm{d}u \\
		\shortintertext{plugging back \(v = u + K^2 (C \sqrt{\frac{n}{m}} + C^2 \frac{n}{m})\),}
		 & \leq K^2 \left( C \sqrt{\frac{n}{m}} + C^2 \frac{n}{m} \right) + \int_{0}^{\infty} 2 e^{-t^2} \,\mathrm{d}u                                                                                                                                   \\
		 & = K^2 \left( C \sqrt{\frac{n}{m}} + C^2 \frac{n}{m} \right)  + \int_{0}^{\infty} 2 e^{-t^2} K^2 \left( \frac{C}{\sqrt{m} } + \frac{2C^2 \sqrt{n} }{m} + \frac{2C^2}{m}t \right) \,\mathrm{d}t                                                 \\
		 & = K^2 \left( C \sqrt{\frac{n}{m}} + C^2 \frac{n}{m} \right) + K^2 \left( \sqrt{\pi } \left( \frac{C}{\sqrt{m} } + \frac{2C^2 \sqrt{n} }{m} \right) + \frac{2C^2}{m} \right),
	\end{align*}
	which is asymptotically \(\asymp K^2 ( \sqrt{\frac{n}{m}} + \frac{n}{m} )\).
\end{answer}

\begin{problem*}[Exercise 4.6.3]\label{ex4.6.3}
	Deduce from Theorem 4.6.1 the following bounds on the expectation:
	\[
		\sqrt{m} - CK^2 \sqrt{n}
		\leq \mathbb{E}_{}[s_n(A)]
		\leq \mathbb{E}_{}[s_1(A)]
		\leq \sqrt{m} + CK^2 \sqrt{n} .
	\]
\end{problem*}
\begin{answer}
	From Theorem 4.6.1, for any \(t \geq 0\),
	\[
		\sqrt{m} - CK^2 (\sqrt{n} + t)
		\leq s_n(A)
		\leq s_1(A)
		\leq \sqrt{m} + CK^2 (\sqrt{n} + t)
	\]
	with probability at least \(1 - 2 \exp (-t^2)\). We want to show that
	\[
		\sqrt{m} - CK^2 \sqrt{n}
		\leq \mathbb{E}_{}[s_n(A)]
		\leq \mathbb{E}_{}[s_1(A)]
		\leq \sqrt{m} + CK^2 \sqrt{n}.
	\]
	Consider
	\[
		\xi
		\coloneqq \frac{\max \left( 0, \sqrt{m} - CK^2 \sqrt{n} - s_n (A) , s_1(A) - \sqrt{m} - CK^2 \sqrt{n}  \right) }{CK^2}
		\geq 0,
	\]
	then from the integral identity,
	\[
		\mathbb{E}_{}[\xi ]
		= \int_{0}^{\infty} \mathbb{P} (\xi > t) \,\mathrm{d}t
		\leq \int_{0}^{\infty} 2 e^{-t^2} \,\mathrm{d}t
		= \sqrt{\pi },
	\]
	which proves the result.
\end{answer}

\begin{problem*}[Exercise 4.6.4]\label{ex4.6.4}
	Give a simpler proof of Theorem 4.6.1, using Theorem 3.1.1 to obtain a concentration bound for \(\lVert Ax \rVert _2\) and \hyperref[ex4.4.4]{Exercise 4.4.4} to reduce to a union bound over a net.
\end{problem*}
\begin{answer}
	From the proof of Theorem 4.6.1, we know that \(S^{n-1}\) admits a \(1 / 4\)-net \(\mathcal{N} \) such that \(\lvert \mathcal{N}  \rvert \leq 9^n\). Furthermore, for any \(x \in \mathcal{N} \), we have
	\begin{itemize}
		\item \(\mathbb{E}_{}[\langle A_i, x \rangle ] = \langle \mathbb{E}_{}[A_i] , x \rangle = \langle 0, x \rangle = 0\);
		\item \(\mathbb{E}_{}[\langle A_i, x \rangle ^2] = x^{\top} \mathbb{E}_{}[A_i ^{\top} A_i] x = x^{\top} I_n x = 1\) (\(x \in S^{n-1}\) too);
		\item \(\lVert \langle A_i, x \rangle  \rVert _{\psi _2} \leq \lVert A_i \rVert _{\psi _2} \leq K\) for all \(i\),
	\end{itemize}
	by Theorem 3.1.1, we have \(\left\lVert \lVert Ax \rVert _2 - \sqrt{m} \right\rVert _{\psi _2} \leq CK^2\). From Proposition 2.5.2 (i), for any \(t > 0\),
	\[
		\begin{split}
			 & \mathbb{P} \left( \lvert \lVert Ax \rVert _2 - \sqrt{m} \rvert > CK (\sqrt{n \log 9} + t) \right) \\
			 & \leq 2\exp (- (\sqrt{n \log 9} + t)^2)
			\leq 2 \exp (- (n \log 9 + t^2))
			= 2 \cdot 9^{-n} \cdot e^{-t^2}.
		\end{split}
	\]
	Finally, from \hyperref[ex4.4.4]{Exercise 4.4.4}, with a union bound over \(\mathcal{N} \), we have
	\begin{align*}
		 & \mathbb{P} \left( \lnot \left\{ \sqrt{m} - 2CK^2 (\sqrt{n \log 9} + t) \leq s_n(A) \leq s_1(A) \leq \sqrt{m} + 2CK^2(\sqrt{n \log 9} + t )\right\}  \right) \\
		\shortintertext{by the definition of \(s_n(A)\) and \(s_1(A)\), we have}
		 & \leq \mathbb{P} \left( \max _{x \in S^{n-1}} \left\lvert \lVert Ax \rVert _2 - \sqrt{m} \right\rvert > 2CK^2 (\sqrt{n \log 9} + t ) \right)                 \\
		 & \leq \mathbb{P} \left( 2 \max _{x \in \mathcal{N} } \left\lvert \lVert Ax \rVert _2 - \sqrt{m} \right\rvert > 2CK^2 (\sqrt{n \log 9} + t) \right)           \\
		 & \leq \sum_{x \in \mathcal{N} } \mathbb{P} \left( \left\lvert \lVert Ax \rVert _2 - \sqrt{m} \right\rvert > CK^2 ( \sqrt{n \log 9} + t) \right)              \\
		 & \leq 9^n \cdot 2 \cdot 9^{-n} \cdot e^{-t^2}
		= 2e^{-t^2}.
	\end{align*}
	Scaling \(C\) accommodates the additional \(\log 9\) factor finishes the proof.
\end{answer}

\section{Application: covariance estimation and clustering}
\begin{problem*}[Exercise 4.7.3]\label{ex4.7.3}
	Our argument also implies the following high-probability guarantee. Check that for any \(u \geq 0\), we have
	\[
		\lVert \Sigma _m - \Sigma  \rVert
		\leq CK^2 \left( \sqrt{\frac{n + u}{m}} + \frac{n+u}{m} \right) \lVert \Sigma \rVert
	\]
	with probability at least \(1 - 2 e^{-u}\).
\end{problem*}
\begin{answer}
	Omit
\end{answer}

\begin{problem*}[Exercise 4.7.6]\label{ex4.7.6}
	Prove Theorem 4.7.5 for the spectral clustering algorithm applied for the Gaussian mixture model. Proceed as follows.
	\begin{enumerate}[(a)]
		\item Compute the covariance matrix \(\Sigma \) of \(X\); note that the eigenvector corresponding to the largest eigenvalue is parallel to \(\mu \).
		\item Use results about covariance estimation to show that the sample covariance matrix \(\Sigma _m\) is close to \(\Sigma \), if the sample size \(m\) is relatively large.
		\item Use the Davis-Kahan Theorem 4.5.5 to deduce that the first eigenvector \(v = v_1(\Sigma _m)\) is close to the direction of \(\mu \).
		\item Conclude that the signs of \(\langle \mu , X_i \rangle \) predict well which community \(X_i\) belongs to.
		\item Since \(v \approx \mu \), conclude the same for \(v\).
	\end{enumerate}
\end{problem*}
\begin{answer}
	Omit
\end{answer}