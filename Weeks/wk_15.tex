\week{15}{8 Jun.\ 2024}{Stochastic Block Model and Community Detection}
\section{Application: community detection in networks}
\begin{problem*}[Exercise 4.5.2]\label{ex4.5.2}
	Check that the matrix \(D\) has rank \(2\), and the non-zero eigenvalues \(\lambda _i\) and the corresponding eigenvectors \(u_i\) are
	\[
		\lambda _1 = \left( \frac{p+q}{2} \right) n, \quad
		u_1 = \begin{bmatrix}
			\mathbbm{1}_{n / 2 \times 1} \\
			\mathbbm{1}_{n / 2 \times 1} \\
		\end{bmatrix} ,\quad
		\lambda _2 = \left( \frac{p-q}{2} \right) n,\quad
		u_2 = \begin{bmatrix}
			\mathbbm{1}_{n / 2 \times 1}   \\
			- \mathbbm{1}_{n / 2 \times 1} \\
		\end{bmatrix}.
	\]
\end{problem*}
\begin{answer}
	Let \(n\) be an even number. Firstly, for any \(D \in \mathbb{R} ^{n \times n}\), columns \(1\) to \(n / 2\) are identical, same for columns \(n / 2 + 1\) to \(n\). Furthermore, since \(p > q\), column \(1\) and \(n / 2 + 1\) are linear independent, so \(\rank (D) = 2\).

	Instead of solving the characteristic equation and find the eigenvalues, and find the corresponding eigenvectors later, since we know that \(\rank (D) = 2\), it's immediate that there are only \(2\) non-zero eigenvalues. Hence, we directly verify that
	\[
		\lambda _1 = \left( \frac{p + q}{2} \right) n, \quad
		u_1 = \mathbbm{1}_{n \times 1}, \quad
		\lambda _2 = \left( \frac{p-q}{2} \right) n, \quad
		u_2 = \begin{pmatrix}
			\mathbbm{1}_{1 \times n / 2}   \\
			- \mathbbm{1}_{1 \times n / 2} \\
		\end{pmatrix}^{\top}.
	\]
	For \(\lambda _1\), indeed, since
	\[
		D u_1
		= \lambda _1 u_1
		\implies \begin{pmatrix}
			\frac{p+q}{2}n \\
			\frac{p+q}{2}n \\
			\vdots         \\
			\frac{q+p}{2}n \\
			\frac{q+p}{2}n \\
		\end{pmatrix}
		= \left( \frac{p+q}{2} \right) n \cdot \begin{pmatrix}
			1      \\
			1      \\
			\vdots \\
			1      \\
			1      \\
		\end{pmatrix}.
	\]
	On the other hand, for \(\lambda _2\), we have
	\[
		D u_2
		= \lambda _2 u_2
		\implies \begin{pmatrix}
			\frac{p-q}{2}n \\
			\frac{p-q}{2}n \\
			\vdots         \\
			\frac{q-p}{2}n \\
			\frac{q-p}{2}n \\
		\end{pmatrix}
		= \left( \frac{p-q}{2} \right) n \cdot \begin{pmatrix}
			1      \\
			1      \\
			\vdots \\
			-1     \\
			-1     \\
		\end{pmatrix},
	\]
	which again holds.
\end{answer}

\begin{problem*}[Exercise 4.5.4]\label{ex4.5.4}
	Deduce Weyl's inequality from the Courant-Fisher's min-max characterization of eigenvalues.
\end{problem*}
\begin{answer}
	We have that from the Courant-Fisher's min-max characterization,
	\[
		\lambda _i (A)
		= \max _{\dim E = i} \min _{x \in S(E)} \langle Ax, x \rangle .
	\]
	Now, as \(\lambda _i (A) = - \lambda _{n-i+1} (-A)\), we see that
	\[
		\lambda _i(A)
		= - \lambda _{n-i+1}(-A)
		= - \max _{\dim E = n-i+1} \min _{x \in S(E)} \langle -Ax, x \rangle
		= \min _{\dim E = n-i+1} \max _{x \in S(E)} \langle Ax, x \rangle.
	\]
	We now show the \hyperref[thm:Weyl-inequality]{Weyl's inequality}.

	\begin{theorem}[Weyl's inequality]\label{thm:Weyl-inequality}
		\(\lambda _{i+j-1}(A+B) \leq \lambda _i(A) + \lambda _j(B) \leq \lambda _{i+j-n}(A+B)\).
	\end{theorem}
	\begin{proof}
		We first show the lower-bound. From the Courant-Fisher's min-max characterization, it suffices to show that for any \(E\) with \(\dim E = i+j-1\), there exists some \(x \in S(E)\) such that \(\langle (A+B)x, x \rangle \leq \lambda _i(A) + \lambda _j(B)\).

		We first analyze \(\lambda _i(A)\). We know that from the max-min characterization,
		\[
			\lambda _i(A)
			= \min _{\dim E = n-i+1} \max _{x \in S(E)} \langle Ax, x \rangle ,
		\]
		i.e., there exists some \(E_A\) with \(\dim E_A = n-i+1\) such that \(\lambda _i(A) = \max _{x \in S(E_A)} \langle Ax, x \rangle \). Similarly, there exists some \(E_B\) with \(\dim E_B = n-j+1\) satisfying the same property. Hence, it suffices to find some unit vector \(x\) in \(E_A \cap E_B \cap E\). We see that
		\[
			\dim (E_A \cap E_B)
			\geq \dim E_A + \dim E_B - n
			= n-i-j+2,
		\]
		which implies that \(E_A \cap E_B\) will have a non-trivial intersection with \(E\) since \(\dim E = i+j-1\), hence we're done. For the upper-bound, taking the negative gives the result.
	\end{proof}

	To obtain the spectral stability, we see that from \hyperref[thm:Weyl-inequality]{Weyl's inequality}, we have
	\[
		\begin{dcases}
			\lambda _i(A+B) \leq \lambda _i(A) + \lambda _1(B) ; \\
			\lambda _i(A+B) \geq \lambda _i(A) + \lambda _n(B) ;
		\end{dcases}
		\implies \lambda _n(B) \leq \lambda _i(A+B) - \lambda _i(A) \leq \lambda _1(B).
	\]
	Given any symmetric \(S, T\), by setting \(A \coloneqq T\) and \(B \coloneqq S - T\), the upper-bound yields
	\[
		\lambda _i(S) - \lambda _i(T)
		\leq \lambda _1(S - T)
		= \lVert S - T \rVert.
	\]
	On the other hand, by setting \(A \coloneqq S\) and \(B\coloneqq T - S\), the upper-bound again yields
	\[
		\lambda _i(T) - \lambda _i(S)
		\leq \lambda _1(T - S)
		= \lVert T - S \rVert
		= \lVert S - T \rVert.
	\]
	As this holds for every \(i\), we have
	\[
		\max _i \lvert \lambda _i(S) - \lambda _i(T) \rvert
		\leq \lVert S - T \rVert
	\]
	as we desired.
\end{answer}